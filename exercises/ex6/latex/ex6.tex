\input{../../../common/preamble.tex}
\input{../../../common/preamble-addon-listings.tex}
\input{../../../common/preamble-addon-figures.tex}

% Figure related stuff
\usepackage{etex}
\usepackage{tikz,pgfplots}
\pgfplotsset{compat=1.9}
\usetikzlibrary{arrows,decorations.markings}


\author{Einar Baumann}
\title{
    \vspace{-1in}
    \usefont{OT1}{bch}{b}{n}
    \vspace{0.1in}
    \rule{\textwidth}{0.5pt} \\[0.5cm]
    \normalfont \normalsize \textsc{TKT4140 Numeriske beregninger med datalab} \\ [20pt]
    {\textsc{ \huge Øving 6 }} \\ [0.5cm]
    {\textsc {\Large Sirkulært innspent plate med konsentrert enkelt last} } \\
    \vspace{0.1in}
    \rule{\textwidth}{2pt} \\[0.7cm]
}

\begin{document}
\maketitle
\thispagestyle{empty}
\clearpage

\noindent Differensialligning for utbøyning:
\begin{equation}
  \frac{d^3 w}{dr^3} + \frac{1}{r} \frac{d^2 w}{dr^2} - \frac{1}{r^2} \frac{dw}{dr} = \frac{8}{r} \, , \; 0 < r < 1 \label{eq:main}
\end{equation}
Randbetingelser:
\begin{equation}
  w(1) = \frac{dw}{dr}(1) = 0 \, , \; \frac{fw}{dr}(0) = 0 \tag{b}
\end{equation}
Analytisk løsning:
\begin{subequations}
\begin{equation}
  w(r) = r^2 \left[ 2 \ln(r) - 1 \right] + 1 \label{eq:analytical_w}
\end{equation}
\begin{equation}
  \frac{dw}{dr}(r) = 4 r \ln (r) \label{eq:analytical_dw}
\end{equation}
\end{subequations}


\section{Diskretisering av helningen $\phi$} % (fold)
\label{sec:diskretisering_av_helningen_phi_}
Innfører
\begin{equation}
  \phi(r) = - \frac{dw}{dr}
\end{equation}
i ligning \eqref{eq:main} og får:
\begin{equation}
  \frac{d^2 \phi}{dr^2} + \frac{1}{r} \frac{d \phi}{dr} - \frac{\phi}{r^2} = -\frac{8}{r} \label{eq:main_med_helning}
\end{equation}
med randbetingelser
\begin{equation}
  \phi(0) = \phi(1) = 0 \tag{b}
\end{equation}

\noindent Deler intervallet $x=[0,1]$ inn i $N+1$ deler med $h \equiv \Delta r = \frac{1}{N+1}$ som illuistrert i Figur~\ref{fig:diskretisering}, og diskretiserer ligning \eqref{eq:main_med_helning} med sentraldifferanser.

\begin{figure}[htbp]
  \centering
  \includegraphics[width=0.95\textwidth]{illustrations/intervaller.pdf}
  \caption{}
  \label{fig:diskretisering}
\end{figure}

\noindent Diskretiserer de deriverte leddene i ligning \eqref{eq:main_med_helning}:
\begin{subequations}
\begin{equation}
  \frac{d^2 \phi}{dr^2}\bigg|_i \approx \frac{\phi_{i-1} - 2\phi_i + \phi_{i+1}}{h^2}
\end{equation}
\begin{equation}
  \frac{d \phi}{dr} \bigg|_i \approx \frac{\phi_{i+1} - \phi_{i-1}}{2h}
\end{equation}
\end{subequations}
setter de inn:
\begin{equation}
  \frac{\phi_{i-1} - 2\phi_i + \phi_{i+1}}{h^2} + \frac{1}{r_i} \frac{\phi_{i+1} - \phi_{i-1}}{2h} - \frac{\phi_i}{r_i^2} = -\frac{8}{r_i} \label{eq:diskretisert_main}
\end{equation}
der $r_i = h\cdot i, i=0,1,\dots,N+1$. Omformer \eqref{eq:diskretisert_main} og multipliserer deretter med $h^2$ før videre forenkling:
\begin{equation}
  -\frac{2\phi_i}{h^2} + \frac{\phi_{i-1}}{h^2} + \frac{\phi_{i+1}}{h^2} + \frac{\phi_{i+1}}{2h^2 i} -  \frac{\phi_{i-1}}{2h^2 i} - \frac{\phi_i}{h^2 i^2} = -\frac{8}{h\cdot i} \nonumber
\end{equation}
\begin{equation}
  -2\phi_i + \phi_{i-1} + \phi_{i+1} + \frac{\phi_{i+1}}{2i} -  \frac{\phi_{i-1}}{2 i} - \frac{\phi_i}{i^2} = -\frac{8h}{i} \nonumber
\end{equation}
\begin{equation}
  \phi_{i-1} \left( 1- \frac{1}{2i} \right) + \\
  \phi_{i}   \left( -2 - \frac{1}{i^2}  \right) + \\
  \phi_{i+1} \left( 1 + \frac{1}{2i} \right) = \\
  -\frac{8h}{i}
\end{equation}

\noindent
Uttrykkene for diagonalene $a_i, b_i, c_i$ og høyresiden $d_i$ blir da
\begin{subequations}
\begin{align}
  a_i &= 1- \frac{1}{2i}   \\
  b_i &= -2 - \frac{1}{i^2} \\
  c_i &= 1 + \frac{1}{2i}   \\
  d_i &= -\frac{8h}{i}
\end{align}
\end{subequations}
% section diskretisering_av_helningen_phi_ (end)




\section{Løsing av ligningssystemet} % (fold)
\label{sec:l_sing_av_ligningssystemet}
Fra randbetingelsene har vi at $\phi(0)=\phi_0=0, \phi(1)=\phi_4=0$. Matrisen vår blir
\begin{equation}
  \begin{bmatrix}
    b_1 & c_1 & 0   \\
    a_2 & b_2 & c_2 \\
    0   & a_3 & b_3 \\
  \end{bmatrix}
  \cdot
  \begin{bmatrix}
    \phi_1 \\ \phi_2 \\ \phi_3
  \end{bmatrix}
  =
  \begin{bmatrix}
    d_1 \\ d_2 \\ d_3
  \end{bmatrix}
\end{equation}
\begin{equation}
  \begin{bmatrix}
    -3          &  \frac{3}{2} & 0 \\
    \frac{3}{4} & -\frac{9}{4} &  \frac{5}{4} \\
    0           & \frac{5}{6}  & -\frac{19}{9} \\
  \end{bmatrix}
  \cdot
  \begin{bmatrix}
    \phi_1 \\ \phi_2 \\ \phi_3
  \end{bmatrix}
  =
  \begin{bmatrix}
    -2 \\ -1 \\ -\frac{2}{3}
  \end{bmatrix}
\end{equation}
som har løsning
\begin{align}
  \phi_1 &= 1.35238 \nonumber \\
  \phi_2 &= 1.37143 \nonumber\\
  \phi_3 &= 0.85714 \nonumber
\end{align}

% section l_sing_av_ligningssystemet (end)




\clearpage
\section{Integrering med Heuns metode} % (fold)
\label{sec:integrering_med_heuns_metode}
Heuns metode reduseres til trapesmetoden:
\begin{equation}
  w_{i+1} = w_i + \frac{h}{2} \left[ \phi_i + \phi_{i+1} \right]
\end{equation}
Bruker her ``motsatt'' nummerering av den vi brukte i forige oppgave, ettersom vi kjenner $w(1)$ men ikke $w(0)$. Dermed er $w(0)=w_4$ og $w(1)=w_0$. Dette gjelder også for $\phi$.

\includegraphics[width=0.95\textwidth]{illustrations/intervaller_inv.pdf}

\begin{center}
\begin{tabular}{l|ll}
\toprule
$n$ & $w_n$ & $\phi_n$ \\
\midrule
0   & 0      & 0      \\
1   & 0.1071 & 0.8571 \\
2   & 0.3857 & 1.3714 \\
3   & 0.7262 & 1.3524 \\
4   & 0.8952 & 0      \\
\bottomrule
\end{tabular}
\end{center}
Og ``oversatt'' tilbake til det tidligere brukte nummereringssystemet:
\begin{equation}
  w_0 = 0.8952, \; w_1 = 0.7262, \; w_2 = 0.3857, \; w_3 = 0.1071 \nonumber
\end{equation}

% section integrering_med_heuns_metode (end)



\section{Diskretisering med ligning (1.2.10)} % (fold)
\label{sec:diskretisering_med_ligning_1_2_10}
Skal nå skrive om
\begin{equation}
  \frac{d}{dr} \left[ \frac{1}{r} \frac{d}{dr} \left[ r\cdot \phi(r) \right] \right] = -\frac{8}{r}
\end{equation}
Med Ligning (1.2.10) fra kompendiet. Ligning (1.2.10) er som følger:
\begin{subequations}
\begin{equation}
  \frac{d}{dx} \left[ p(x) \cdot \frac{d}{dx}(u(x)) \right]\bigg|_i \approx
  \frac{p_{i-\frac{1}{2}} \cdot u_{i-1} - \left( p_{i+\frac{1}{2}} + p_{i-\frac{1}{2}} \right)
  \cdot u_i + p_{i + \frac{1}{2}} \cdot u_{i+1} }{h^2} + \text{feilledd}
\end{equation}
\begin{equation}
  \text{feilledd} = - \frac{h^2}{24} \cdot \frac{d}{dx} \left( p(x) \cdot u'''(x) + \left[ p(x) \cdot u'(x) \right]'' \right) + O(h^3)
\end{equation}
\end{subequations}
I dette tilfelle har vi at
\begin{align}
  \nonumber
  p(x) &= \frac{1}{r}, \;\; p_i = \frac{1}{r_i} = \frac{1}{h\cdot i} \\
  \nonumber
  u(x) &= r\cdot \phi(r), \;\; u_i = r_i \phi_i = h\cdot i \cdot \phi_i
\end{align}
Det diskretiserte systemet blir da (når man ser bort fra feilledet):
\begin{multline*}
    \frac{d}{dr} \left[ \frac{1}{r} \frac{d}{dr} \left[ r\cdot \phi(r) \right] \right] \bigg|_i
    \approx
    \frac{
      \frac{h\cdot(i-1)}{h\cdot\left(i-\frac{1}{2}\right)}  \phi_{i-1} -
      \left( \frac{1}{h\cdot \left( i+\frac{1}{2} \right) } +
             \frac{1}{h\cdot \left( i-\frac{1}{2} \right) }
      \right) \cdot h\cdot i \cdot \phi_i +
      \frac{h\cdot (i+1)}{h\cdot \left( i + \frac{1}{2} \right)} \cdot
      \phi_{i+1}
    }
    {h^2} = - \frac{8}{h\cdot i} \\
    \implies
      \left( \frac{i-1}{i-\frac{1}{2}} \right) \cdot \phi_{i-1} -
      \left( \frac{i}{i+\frac{1}{2} } +
             \frac{i}{i-\frac{1}{2} }
      \right) \cdot \phi_i +
      \left( \frac{i+1}{i + \frac{1}{2}} \right) \cdot \phi_{i+1}
    = -\frac{8h}{i}
\end{multline*}

% section diskretisering_med_ligning_1_2_10 (end)


\end{document}
