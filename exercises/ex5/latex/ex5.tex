\documentclass[norsk,11pt,a4paper]{article}
\usepackage[T1]{fontenc} % --------------| More characters.
\usepackage[utf8]{inputenc} % ---------| Direct use of scandinavian letters.
\usepackage{float} % --------------------| More options for floats.
\usepackage{graphicx} % -----------------| Support more image formats.
\usepackage{booktabs} % -----------------| Better-looking tables.
\usepackage{tabularx} % -----------------| Better tables
\usepackage{subfig} % -------------------| Subfigures.
\usepackage[a4paper]{geometry} % --------| Adjusting page margins.
\usepackage{amsmath,amssymb,amsfonts} % -| Various math, including eqref.
\usepackage{color} % --------------------| Allows defn. of custom colors.
\usepackage{babel}

% XY-pic. Used for creating illustrations.
\input xy
\xyoption{all}

% Styling captions.
\usepackage{caption}
\captionsetup{margin=10pt,font=small,labelfont=bf}

% Changing section header and title fonts.
\usepackage{sectsty}
\allsectionsfont{\normalfont\sffamily}


%******************************************************************************
% Includes the listings package and sets some settings for it.
% REQUIRES the `color' package.
%******************************************************************************

\usepackage{listings} % -----------------------| Used for source code listings.
\definecolor{lst-gray}{RGB}{100,100,100}  % ---|Color for line-numbers.
\definecolor{lst-light-gray}{RGB}{250,250,250} % Background color for listings.
\definecolor{lst-dark-green}{RGB}{45,111,0} % -| Dark green for comments.
\lstset{
  aboveskip=0em, % -------------------| Skip above listing box.
  backgroundcolor=\color{lst-light-gray}, % Background color.
  basicstyle=\ttfamily\scriptsize, % -| Default font style.
  belowskip=\topskip, % --------------| Skip below listing box.
  breakatwhitespace=false, % ---------| Automatic breaks only at whitespace?.
  breaklines=true, % -----------------| Sets automatic line breaking.
  captionpos=t, % --------------------| Sets the caption-position to bottom.
  commentstyle=\color{lst-dark-green}, % ------| Comment style.
  escapeinside={\%*}{*)}, % ----------| For adding LaTeX within code.
  frame=single, % --------------------| Adds a frame around the code.
  keepspaces=true, % -----------------| Keeps spaces in text.
  keywordstyle=\color{blue}, %--------| Keyword style.
  language={C}, % --------------------| The language of the code.
  literate={æ}{{\ae}}1 % -------------| Character conversions
           {Æ}{{\AE}}1
           {ø}{{\oe}}1
           {Ø}{{\OE}}1
           {å}{{\aa}}1
           {Å}{{\AA}}1
           {µ}{{\ensuremath{\mu}}}1,
  numbers=left, % --------------------| Line-number position: none/left/right.
  numbersep=5pt, % -------------------| Distance between line-numbers and code.
  numberstyle=\tiny\color{lst-gray}, %| The style that used for line-numbers.
  rulecolor=\color{black}, % ---------| Frame color.
  showspaces=false, % ----------------| Show spaces with underscores.
  showstringspaces=false, % ----------| Underline spaces within strings.
  showtabs=false, % ------------------| Show tabs with underscores.
  stepnumber=1, %---------------------| Step between two line-numbers..
  stringstyle=\color{red}, % ---------| String literal style.
  tabsize=4, % -----------------------| Sets default tabsize to 2 spaces.
  title=\lstname % -------------------| Show the filename of included file.
}
%******************************************************************************
% This preabmle contains packages needed to create figures and plots in LaTeX.
%******************************************************************************

\usepackage{etex}
\usepackage{tikz,pgfplots}
\pgfplotsset{compat=1.9}
\usetikzlibrary{arrows,decorations.markings}
\usetikzlibrary{calc}

% Vector Styles for drawings
\tikzstyle{load}   = [ultra thick,-latex]
\tikzstyle{stress} = [-latex]
\tikzstyle{dim}    = [latex-latex]
\tikzstyle{axis}   = [-latex,black!55]


% Figure related stuff
\usepackage{etex}
\usepackage{tikz,pgfplots}
\pgfplotsset{compat=1.9}
\usetikzlibrary{arrows,decorations.markings}


\author{Einar Baumann}
\title{
    \vspace{-1in}
    \usefont{OT1}{bch}{b}{n}
    \vspace{0.1in}
    \rule{\textwidth}{0.5pt} \\[0.5cm]
    \normalfont \normalsize \textsc{TKT4140 Numeriske beregninger med datalab} \\ [20pt]
    {\textsc{ \huge Øving 5}} \\
    \vspace{0.1in}
    \rule{\textwidth}{2pt} \\[0.7cm]
}

\begin{document}
\maketitle
\thispagestyle{empty}
\clearpage




\section{Jeffrey-Hamel strømning} % (fold)
\label{sec:jeffrey_hamel_str_mning}
\begin{equation}
  f'''(\eta) + 2 \mathrm{Re} \cdot \alpha \cdot f(\eta) \cdot f'(\eta) + 4 \alpha^2 f'(\eta) = 0
  \label{eq:1main}
\end{equation}
\begin{equation}
  f(0) = 1, \; f'(0) = 0, \; f(1) = 0 \tag{b}
\end{equation}

\subsection{Løsing av \eqref{eq:1main} som initialverdiproblem} % (fold)
\label{sub:l_sing_av_}
Eulerskjemaet for ligning \eqref{eq:1main} er som følger:
\begin{subequations}
  \begin{align}
    f_{n+1}    &= f_n   + \Delta \eta f'_n \\
    f'_{n+1}   &= f'_n  + \Delta \eta f''_n \\
    f''_{n+1}  &= f''n  + \Delta \eta f'''_n \\
    f'''_n{+1} &= - 2 \mathrm{Re} \cdot \alpha \cdot f_{n+1} \cdot f'_{n+1} - 4 \alpha^2 f'_{n+1}
  \end{align}
\end{subequations}

\begin{table}[H]
  \centering
  \label{tab:1euler}
  \begin{tabularx}{0.7\textwidth}{XXXXX}
    \toprule
    $n$ & 0   & 1    & 2    & 3    \\
    $x$ & 0.0 & 0.05 & 0.10 & 0.15 \\
    \midrule
    $f$ & 1.0  &  1.0    &  0.984  &  0.951  \\
    $f'$ & 0.0  &  -0.325 &  -0.65  &  -0.959 \\
    $f''$ & -6.5 &  -6.5   &  -6.187 &  -5.572 \\
    $f'''$ & -0.0 &  6.254  &  12.310 &  17.581 \\
    \bottomrule
  \end{tabularx}
\end{table}
% subsection l_sing_av_ (end)

\subsection{Løsing av \eqref{eq:1main} med skyteteknikk} % (fold)
\label{sub:l_sing_av_eqr}
\textbf{1)} Skriv ligning \eqref{eq:1main} om til et system:
\begin{subequations}
\begin{align}
  y_1 &= f(\eta) \\
  y_2 &= f'(\eta) = y'_1 \\
  y_3 &= f''(\eta) = y'_2 \\
  y_4 &= f'''(\eta) = y'_3 =  2 \mathrm{Re} \cdot \alpha f_1 f_2 - 4 \alpha^2 f_2
\end{align}
\end{subequations}

\noindent \textbf{2)} Bestem $s=y''(0)=y_3(0)$ slik at randbetingelsen $y_1(1)=0$ oppfylles. Setter
\begin{equation}
  \phi(s^m) = y_1(1;s^m) - 1, m=0,1,\dots \nonumber
\end{equation}
slik at
\begin{equation}
   y(1) \rightarrow 0 \text{ for } \phi(s^m) \rightarrow 0, m \rightarrow \infty \nonumber
\end{equation}
Dette kan vi finne ved å bruke en iterasjonsprosess, for eksempel sekantmetoden, til å finne nullpunktene til funksjonen $\phi(s)=0$. Vi må gjette to verdier $s^0, s^1$ før vi kan begynne iterasjonsprosessen.
% subsection l_sing_av_eqr (end)

% section jeffrey_hamel_str_mning (end)


\clearpage
\section{Partikkel langs sirkulær bane} % (fold)
\label{sec:partikkel_langs_sirkul_r_bane}
Bevegelsesligning:
\begin{equation}
  \frac{d^2\theta}{dt^2} = - \mu \left( \frac{d\theta}{dt} \right)^2 + \frac{g}{R} \left( \cos \theta - \mu \sin \theta \right)
\end{equation}
Tolker følgende randbetingelser:
\begin{equation}
  \tag{b}
  \dot\theta(0) = 0, \; \dot\theta(\theta_0) = 0
\end{equation}
For å løse ligningen med skyteteknikk skrives den først om til et system:
\begin{subequations}
\begin{align}
  \theta_1 &= \theta(t) \\
  \theta_2 &= \dot\theta(t) = \dot\theta_1 \\
  \theta_3 &= \ddot\theta(t) = \dot\theta_2
\end{align}
\end{subequations}
Ettersom oppgaven sier at vi vil finne verdien av $\mu$ som fører til at partikkelen stopper ved $\theta=\theta_0$, vil vi her variere $\mu$ i stedet for $s$ inntil randbetingelsene ``treffer''. $\phi(\mu^{(m)})$ funksjonen vår blir da
\begin{equation}
  \phi(\mu^{(m)}) = \theta(\dot\theta=0;\mu^{(m)}) - \theta_0 \label{eq:2phi}
\end{equation}
Vi kan gjette to verdier $\mu^0,\mu^1$ bruke sekantmetoden til å finne roten til \eqref{eq:2phi}, som gir oss den korrekte $\mu$-verdien $\mu = \mu^*$.
% section partikkel_langs_sirkul_r_bane (end)



\end{document}
