\input{../../../common/preamble.tex}
\input{../../../common/preamble-addon-listings.tex}
\usepackage{cancel}

\DeclareMathOperator\erf{erf}

\begin{document}
\author{Einar Baumann}
\title{TKT4140 Øving 4}
\maketitle

\begin{equation}
  f''(\eta) + 2\eta \sqrt{f'(\eta)} = 0
  \label{eq:main}
\end{equation}
\begin{equation*}
  f(0) = 0, \; f(\delta) = 1
\end{equation*}

\pagebreak

\section{Løsing av ligning \eqref{eq:main} som initialverdiproblem med Eulers metode} % (fold)
\label{sec:euler}

Eulers metode:

\begin{align}
  f_{n+1}  &= f_n + \Delta\eta f_n' \label{eq:euler} \\
  f_{n+1}' &= f_n' + \Delta\eta f_n''
\end{align}

\begin{equation}
  f''(\eta) = -2\eta \sqrt{f'(\eta)}
\end{equation}

\begin{center}
  \begin{tabular}{ll|lll}
  \toprule
  $n$ & $\eta$ & $f(\eta)$ & $f'(\eta)$ & $f''(\eta)$ \\
  \midrule
  0   & 0.0    & 0.0000    & 1.2500     & 0.0000      \\
  1   & 0.2    & 0.2500    & 1.2500     & -0.4472     \\
  2   & 0.4    & 0.5000    & 1.1606     & -0.8618     \\
  3   & 0.6    & 0.7321    & 0.9882     & -1.1929     \\
  \bottomrule
  \end{tabular}
\end{center}

% section euler (end)


\section{Omskriving til system} % (fold)
\label{sec:omskriving_til_system}
\begin{align}
  g(\eta)  &= f'(\eta) \\
  g'(\eta) &= f''(\eta) = -2\eta \sqrt{g(\eta)}
\end{align}
\begin{equation*}
  f(0) = 0, \; f(\delta) = 1
\end{equation*}

Vi må bestemme $s = f'(0) = g(0)$ slik at $f(1)$ oppfylles. Vi tipper deretter to verdier $s^0$ og $s^1$, og beregner den ``korrekte'' $s$ vha. lineær interpolering.
% section omskriving_til_system (end)


\section{Skjikttykkelse} % (fold)
\label{sec:skjikttykkelse}

\begin{equation}
  \frac{d}{d\eta}\left[ \sqrt{f'(\eta)} \right] = \frac{f''(\eta)}{2\sqrt{f'(\eta)}}
  \label{eq:diff}
\end{equation}
Omformer og setter \eqref{eq:diff} inn i \eqref{eq:main}:
\begin{align}
  \frac{d}{d\eta}\left[ \sqrt{f'(\eta)} \right] \times 2\sqrt{f'(\eta)} + 2\eta \sqrt{f'(\eta)} &= 0 \nonumber \\
   \frac{d}{d\eta}\left[ \sqrt{f'(\eta)} \right] &= - \eta \nonumber \\
   \sqrt{f'(\eta)} &= -\int \eta d\eta = - \frac{\eta^2}{2} + C \nonumber \\
   f'(\eta) &= \left( - \frac{\eta^2}{2} + C \right)^2
\end{align}

\begin{align}
  f'(\delta) = - \frac{\delta^2}{2} + C &= 0 \nonumber \\
  C &= \frac{\delta^2}{2}
\end{align}

\begin{equation}
  f'(\eta) = \left( \frac{\eta^2}{2} + \frac{\delta^2}{2} \right)^2
\end{equation}

\begin{align}
  f'(0) &= \left( \frac{0}{2} + \frac{\delta^2}{2} \right)^2 \nonumber \\
  \sqrt{f'(0)} &= \frac{\delta^2}{2} \nonumber \\
  2\sqrt{f'(0)} &= \delta^2 \nonumber \\
  \sqrt{2\sqrt{f'(0)}} &= \delta
\end{align}

$\therefore$
% section skjikttykkelse (end)

\section{Eksempel i avsn. 2.2 med nye randbetingelser} % (fold)
\label{sec:eksempel_i_avsn_2_2_med_nye_randbetingelser}
\begin{equation}
  y''(x) = \frac{3}{2} y^2 \label{eq:main2}
\end{equation}
\begin{equation}
  y'(0) = -8, \; y'(1) = -1 \nonumber
\end{equation}

Omskriver til \eqref{eq:main2} til ligningssystem:
\begin{align}
  y_1(x) & = y(x) \\
  y_2(x) & = y_1'(x) \\
  y_3(x) & = y_2'(x) = \frac{3}{2} \left[ y_1(x) \right]^2
\end{align}
\begin{equation}
  y_2(0) = -8, \; y_2(1) = -1 \nonumber
\end{equation}

Vi bestemmer $s = y_1(0)$ slik at randbetingelsen $y_2(1) = -1$ blir oppfylt:
\begin{equation}
  \phi(s^m) = y_2(1,s^m) + 1 = 0, \; m = 0,1,\dots
\end{equation}
Dette gjøres ved å først gjette to verdier $s^0$ og $s^0$, og deretter bruke sekantmetoden til å bestemme skjæringspunktet $\phi(s) = 0$.


% section eksempel_i_avsn_2_2_med_nye_randbetingelser (end)

\end{document}
