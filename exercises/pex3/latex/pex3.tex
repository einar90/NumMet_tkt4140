\documentclass[norsk,11pt,a4paper]{article}
\usepackage[T1]{fontenc} % --------------| More characters.
\usepackage[utf8]{inputenc} % ---------| Direct use of scandinavian letters.
\usepackage{float} % --------------------| More options for floats.
\usepackage{graphicx} % -----------------| Support more image formats.
\usepackage{booktabs} % -----------------| Better-looking tables.
\usepackage{tabularx} % -----------------| Better tables
\usepackage{subfig} % -------------------| Subfigures.
\usepackage[a4paper]{geometry} % --------| Adjusting page margins.
\usepackage{amsmath,amssymb,amsfonts} % -| Various math, including eqref.
\usepackage{color} % --------------------| Allows defn. of custom colors.
\usepackage{babel}

% XY-pic. Used for creating illustrations.
\input xy
\xyoption{all}

% Styling captions.
\usepackage{caption}
\captionsetup{margin=10pt,font=small,labelfont=bf}

% Changing section header and title fonts.
\usepackage{sectsty}
\allsectionsfont{\normalfont\sffamily}


%******************************************************************************
% Includes the listings package and sets some settings for it.
% REQUIRES the `color' package.
%******************************************************************************

\usepackage{listings} % -----------------------| Used for source code listings.
\definecolor{lst-gray}{RGB}{100,100,100}  % ---|Color for line-numbers.
\definecolor{lst-light-gray}{RGB}{250,250,250} % Background color for listings.
\definecolor{lst-dark-green}{RGB}{45,111,0} % -| Dark green for comments.
\lstset{
  aboveskip=0em, % -------------------| Skip above listing box.
  backgroundcolor=\color{lst-light-gray}, % Background color.
  basicstyle=\ttfamily\scriptsize, % -| Default font style.
  belowskip=\topskip, % --------------| Skip below listing box.
  breakatwhitespace=false, % ---------| Automatic breaks only at whitespace?.
  breaklines=true, % -----------------| Sets automatic line breaking.
  captionpos=t, % --------------------| Sets the caption-position to bottom.
  commentstyle=\color{lst-dark-green}, % ------| Comment style.
  escapeinside={\%*}{*)}, % ----------| For adding LaTeX within code.
  frame=single, % --------------------| Adds a frame around the code.
  keepspaces=true, % -----------------| Keeps spaces in text.
  keywordstyle=\color{blue}, %--------| Keyword style.
  language={C}, % --------------------| The language of the code.
  literate={æ}{{\ae}}1 % -------------| Character conversions
           {Æ}{{\AE}}1
           {ø}{{\oe}}1
           {Ø}{{\OE}}1
           {å}{{\aa}}1
           {Å}{{\AA}}1
           {µ}{{\ensuremath{\mu}}}1,
  numbers=left, % --------------------| Line-number position: none/left/right.
  numbersep=5pt, % -------------------| Distance between line-numbers and code.
  numberstyle=\tiny\color{lst-gray}, %| The style that used for line-numbers.
  rulecolor=\color{black}, % ---------| Frame color.
  showspaces=false, % ----------------| Show spaces with underscores.
  showstringspaces=false, % ----------| Underline spaces within strings.
  showtabs=false, % ------------------| Show tabs with underscores.
  stepnumber=1, %---------------------| Step between two line-numbers..
  stringstyle=\color{red}, % ---------| String literal style.
  tabsize=4, % -----------------------| Sets default tabsize to 2 spaces.
  title=\lstname % -------------------| Show the filename of included file.
}
%******************************************************************************
% This preabmle contains packages needed to create figures and plots in LaTeX.
%******************************************************************************

\usepackage{etex}
\usepackage{tikz,pgfplots}
\pgfplotsset{compat=1.9}
\usetikzlibrary{arrows,decorations.markings}
\usetikzlibrary{calc}

% Vector Styles for drawings
\tikzstyle{load}   = [ultra thick,-latex]
\tikzstyle{stress} = [-latex]
\tikzstyle{dim}    = [latex-latex]
\tikzstyle{axis}   = [-latex,black!55]

\lstset{language={Python}}


\author{Einar Baumann}
\title{
    \vspace{-1in}
    \usefont{OT1}{bch}{b}{n}
    \vspace{0.1in}
    \rule{\textwidth}{0.5pt} \\[0.5cm]
    \normalfont \normalsize \textsc{TKT4140 Numeriske beregninger med datalab} \\ [20pt]
    {\textsc{ \huge Pythonøving 3 }} \\ [0.5cm]
    {\textsc {\Large Temperaturfelt i bjelketverrsnitt} } \\
    \vspace{0.1in}
    \rule{\textwidth}{2pt} \\[0.7cm]
}

\begin{document}
\maketitle
\thispagestyle{empty}
\clearpage

\section*{Oppgave 1}
Skal diskretisere og løse den stasjonære varmeledningsligninga:
\begin{equation}
  \frac{\partial^2 T}{\partial x^2} + \frac{\partial^2 T}{\partial y^2} = 0
\end{equation}
Randverdiene er vist i illustrasjonen i Figur~\ref{fig:system}.

\begin{figure}[htbp]
  \centering
  \includegraphics[]{illustrations/system.pdf}
  \caption{Bjelketverrsnitt}
  \label{fig:system}
\end{figure}

Til diskretiseringen brukes 5-punktsformelen (illustrert i Figur~\ref{fig:stencil}). Resultatet av diskretiseringen er som følger:
\begin{equation}
  u_{i-1,j} + u_{i+1,j} + u_{i,j-1} + u_{i,j+1} - 4u_{i,j} = 0
\end{equation}
for
\begin{equation}
\nonumber
  0<i<N_i-1,\;\; 0<j<N_j-1
\end{equation}
der $N_i$ og $N_j$ hhv. er antallet punkter i $x$- og $y$-retning. Utenfor dette er randen, der alle verdiene allerede er kjente. Den lokale $i,j$-indekseringen som er brukt er illustrert i Figur~\ref{fig:indexing}.

\begin{figure}[htbp]
  \centering
  \includegraphics{illustrations/stencil.pdf}
  \caption{Illustrasjon av stensil for fempunktsformelen.}
  \label{fig:stencil}
\end{figure}

\begin{figure}[htbp]
  \centering
  \includegraphics{illustrations/mesh.pdf}
  \caption{Illustrasjon av måten $i,j$-koordinatene er brukt på overflaten. Her er $N_i=N_j=4$, og systemet er null-indeksert.}
  \label{fig:indexing}
\end{figure}

For å få dette inn i en matrise på en god måte bruker vi en global indeksering. ``Mønsteret'' som er valgt er vist i Figur~\ref{fig:numbering}: fra venstre mot høyre.

\begin{figure}[htbp]
  \centering
  \includegraphics{illustrations/numbering.pdf}
  \caption{Illustrasjon av ligningsnumereringen som er brukt.}
  \label{fig:numbering}
\end{figure}

Setter temperaturen i hjørnene lik  gjennomsnittet av temperaturen på kantene inn mot hjørnet.

Løsningsmetoden jeg har brukt benytter seg av Kroneckerproduktet $A\otimes B$ av to matriser (også kalt tensorproduktet), og er beskrevet i seksjon A.5 i \cite{strang}. Skrevet som en blokkmatrise er Kroneckerproduktet som følger:
\begin{equation}
  A \otimes B =
  \begin{bmatrix}
    a_11 B & a_12 B & \cdots & a_1n B \\
    a_21 B & a_22 B & \cdots & a_2n B \\
    \cdot  & \cdot  & \cdots & \cdot  \\
    a_m1 B & a_m2 B & \cdots & a_mn B
  \end{bmatrix}
\end{equation}
Her bruker jeg Kroneckerproduktet til å lage en 2D differansematrise for fempunktsformelen ut fra 1D differansematrisen for trepunktsformelen langs én rad. Matrisene programmet mitt genererer for $n=4$ punkter er som følger:

\begin{align}
   \text{1D Differansematrise, én linje: } A &=
   \begin{bmatrix}
      2 & -1 &  0 &  0 \\
     -1 &  2 & -1 &  0 \\
      0 & -1 &  2 & -1 \\
      0 &  0 & -1 &  2
   \end{bmatrix} \\
   \text{Identitetsmatrise: } I &=
   \begin{bmatrix}
     1 & 0 & 0 & 0 \\
     0 & 1 & 0 & 0 \\
     0 & 0 & 1 & 0 \\
     0 & 0 & 0 & 1
   \end{bmatrix} \\
 \end{align}
Et utdrag av 2D differansematrisen $T$ som brukes videre er som følger (den er rimelig stor, $16\times 16$):
\begin{equation}
  T = (A \otimes I) + (I \otimes A) =
  \begin{bmatrix}
      4 & -1 &  0 &  0 & -1 & -0 &  0 &  0 &  0 & \cdots  0 \\
     -1 &  4 & -1 &  0 & -0 & -1 & -0 &  0 &  0 & \cdots  0 \\
        &    &    &    &    & \cdot &    &    &    &        \\
      0 & -0 & -1 & -0 &  0 & -1 &  4 & -1 &  0 & \cdots  0 \\
      0 &  0 & -0 & -1 &  0 &  0 & -1 &  4 &  0 & \cdots  0 \\
      0 &  0 &  0 &  0 & -1 & -0 &  0 &  0 &  4 & \cdots  0 \\
      0 &  0 &  0 &  0 & -0 & -1 & -0 &  0 & -1 & \cdots  0 \\
        &    &    &    &    & \cdot &    &    &    &        \\
      0 &  0 &  0 &  0 &  0 &  0 &  0 &  0 &  0 & \cdots -1 \\
      0 &  0 &  0 &  0 &  0 &  0 &  0 &  0 &  0 & \cdots  4
  \end{bmatrix}
  \label{eq:diffmat}
\end{equation}
Merk at dette er beregnet med funksjonen \texttt{numpy.kron(A,B)}.

For å kunne anvende matrisen i ligning \eqref{eq:diffmat} på vårt problem må vi gjøre noen tilpasninger for å ta høyde for randverdiene våre.

Matrisen inneholder i utgangspunktet én linje pr. punkt i rutenettet vårt. Ettersom verdiene er kjent langs randen vil ikke lignene vi får ut av matrisen holde i disse punktene. For linjer som svarer til punkter med kjente verdier (randverdier) setter jeg derfor alle elementer unntatt diagonalelementet til 0, og diagonalelementet til 1. Setter i tillegg elementet i $b$ (vektoren på høyre side) lik den kjente verdien i punktet. I koden gjøres dette i følgende løkke:
\begin{lstlisting}
for i in range(n):
    for j in range(n):
        if i == 0 or j == 0 or i == (n-1) or j == (n-1):
            b[i*n+j] = boundaries[i][j]
            T[i*n+j] = np.zeros(n*n)
            T[i*n+j][i*n+j] = 1
\end{lstlisting}

Vi kan nå sette opp systemet $Tu=b$, som enkelt løses med \texttt{numpy.linalg.solve(T,b)}. Etter dette gjør jeg den resulterende $u$-vektoren om til en 2D matrise og plotter den vha. \texttt{contourf}. Resultatet av dette er vist i Figur~\ref{fig:temps}. Den fullstendige koden er vist i Listing~\ref{lst:program} sist i dokumentet.


\begin{figure}[htbp]
  \centering
  \includegraphics[width=0.95\textwidth]{../code/temps.pdf}
  \caption{Plott av den beregnede temperaturen.}
  \label{fig:temps}
\end{figure}


\begin{thebibliography}{1}
  \bibitem{strang}
    Gilbert Strang,
    \emph{Linear Algebra and Its Applications}.
    Cengage Learning,
    4th Edition,
    2005.

\end{thebibliography}


\clearpage
\lstinputlisting[%
  caption={tensorlaplace.py},
  label={lst:program},
  language={Python}]
  {../code/tensorlaplace.py}


\end{document}
