\input{../../../common/preamble.tex}
\input{../../../common/preamble-addon-listings.tex}
\input{../../../common/preamble-addon-figures.tex}
\input{../../../common/preamble-addon-envs.tex}

\author{Einar Baumann}
\title{
    \vspace{-1in}
    \usefont{OT1}{bch}{b}{n}
    \vspace{0.1in}
    \rule{\textwidth}{0.5pt} \\[0.5cm]
    \normalfont \normalsize \textsc{TKT4140 Numeriske beregninger med datalab} \\ [20pt]
    {\textsc{ \huge Pythonøving 2}} \\ [0.5cm]
    {\textsc {\Large Satellittbanar rundt ein planet} } \\
    \vspace{0.1in}
    \rule{\textwidth}{2pt} \\[0.7cm]
}

\begin{document}
\maketitle
\thispagestyle{empty}
\clearpage

\noindent Newtons gravitasjonslov:
\begin{equation}
  \mathbf{F} = -G \frac{Mm}{r^3} \mathbf{r}
\end{equation}
der
\begin{equation}
  \nonumber
  \mathbf{F} = [F_x,F_y] = [m\ddot x,m\ddot y], \;\; \mathbf{r}=[x,y], \;\; r = \sqrt{x^2+y^2}
\end{equation}
slik at
\begin{subequations}
\begin{equation}
  \ddot x = -G \frac{Mx}{\left( \sqrt{x^2+y^2} \right)^3} \label{eq:grav_x}
\end{equation}
\begin{equation}
  \ddot y = -G \frac{My}{\left( \sqrt{x^2+y^2} \right)^3} \label{eq:grav_y}
\end{equation}
\end{subequations}
med initialverdier
\begin{equation}
  \tag{init}
  x(t_0) = x_0, \;\; y(y_0) = y_0, \;\; u(t_0) = u_0, \;\; v(t_0) = v_0
\end{equation}

\section{Omskriving til ligningssystem} % (fold)
\label{sec:omskriving_til_ligningssystem}
Skriver om ligning \eqref{eq:grav_x} og \eqref{eq:grav_y} til et system av førsteordens ligninger:
\begin{align*}
  \frac{dx}{dt} &= u  \\
  \frac{dy}{dt} &= v  \\
  \frac{du}{dt} &= -G \frac{Mx}{\left( \sqrt{x^2+y^2} \right)^3}  \\
  \frac{dv}{dt} &= -G \frac{My}{\left( \sqrt{x^2+y^2} \right)^3}
\end{align*}
Innfører følgende notasjon:
\begin{equation*}
  x = g_0, \;\; y = g_1, \;\; u = g_2, \;\; v = g_3
\end{equation*}
Ligningssystemet ved tid $n$ blir da
\begin{subequations}
\begin{align}
  \dot g_{0,n} &= g_{2,n}  \\
  \dot g_{1,n} &= g_{3,n}  \\
  \dot g_{2,n} &= -G \frac{Mg_{0,n}}{\left( \sqrt{g_{0,n}^2+g_{1,n}^2} \right)^3}  \\
  \dot g_{3,n} &= -G \frac{Mg_{1,n}}{\left( \sqrt{g_{0,n}^2+g_{1,n}^2} \right)^3}
\end{align}
\end{subequations}

\begin{equation}
  \mathbf{\dot g_n} = \mathbf{f}(g_{0,n}, g_{1,n}, g_{2,n}, g_{3,n})
\end{equation}

% section omskriving_til_ligningssystem (end)



\section{Løsing med Runge-Kutta-skjemaet av 4. orden} % (fold)
\label{sec:l_sing_med_runge_kutta_skjemaet_av_4_orden}
RK4 skjemaet for $x$-komponentene, sammen med ligningen som beskriver posisjonen, er som følger ($\Delta t = h$):
\begin{subequations}
\begin{eqnarray}
  k_1 &= & \\
  k_2 &= &
\end{eqnarray}
\end{subequations}

% section l_sing_med_runge_kutta_skjemaet_av_4_orden (end)


\section{Løsing med RK4 fra odespy} % (fold)
\label{sec:l_sing_med_rk4_fra_odespy}

% section l_sing_med_rk4_fra_odespy (end)

\end{document}
