\documentclass[norsk,11pt,a4paper]{article}
\usepackage[T1]{fontenc} % --------------| More characters.
\usepackage[utf8]{inputenc} % ---------| Direct use of scandinavian letters.
\usepackage{float} % --------------------| More options for floats.
\usepackage{graphicx} % -----------------| Support more image formats.
\usepackage{booktabs} % -----------------| Better-looking tables.
\usepackage{tabularx} % -----------------| Better tables
\usepackage{subfig} % -------------------| Subfigures.
\usepackage[a4paper]{geometry} % --------| Adjusting page margins.
\usepackage{amsmath,amssymb,amsfonts} % -| Various math, including eqref.
\usepackage{color} % --------------------| Allows defn. of custom colors.
\usepackage{babel}

% XY-pic. Used for creating illustrations.
\input xy
\xyoption{all}

% Styling captions.
\usepackage{caption}
\captionsetup{margin=10pt,font=small,labelfont=bf}

% Changing section header and title fonts.
\usepackage{sectsty}
\allsectionsfont{\normalfont\sffamily}


%******************************************************************************
% Includes the listings package and sets some settings for it.
% REQUIRES the `color' package.
%******************************************************************************

\usepackage{listings} % -----------------------| Used for source code listings.
\definecolor{lst-gray}{RGB}{100,100,100}  % ---|Color for line-numbers.
\definecolor{lst-light-gray}{RGB}{250,250,250} % Background color for listings.
\definecolor{lst-dark-green}{RGB}{45,111,0} % -| Dark green for comments.
\lstset{
  aboveskip=0em, % -------------------| Skip above listing box.
  backgroundcolor=\color{lst-light-gray}, % Background color.
  basicstyle=\ttfamily\scriptsize, % -| Default font style.
  belowskip=\topskip, % --------------| Skip below listing box.
  breakatwhitespace=false, % ---------| Automatic breaks only at whitespace?.
  breaklines=true, % -----------------| Sets automatic line breaking.
  captionpos=t, % --------------------| Sets the caption-position to bottom.
  commentstyle=\color{lst-dark-green}, % ------| Comment style.
  escapeinside={\%*}{*)}, % ----------| For adding LaTeX within code.
  frame=single, % --------------------| Adds a frame around the code.
  keepspaces=true, % -----------------| Keeps spaces in text.
  keywordstyle=\color{blue}, %--------| Keyword style.
  language={C}, % --------------------| The language of the code.
  literate={æ}{{\ae}}1 % -------------| Character conversions
           {Æ}{{\AE}}1
           {ø}{{\oe}}1
           {Ø}{{\OE}}1
           {å}{{\aa}}1
           {Å}{{\AA}}1
           {µ}{{\ensuremath{\mu}}}1,
  numbers=left, % --------------------| Line-number position: none/left/right.
  numbersep=5pt, % -------------------| Distance between line-numbers and code.
  numberstyle=\tiny\color{lst-gray}, %| The style that used for line-numbers.
  rulecolor=\color{black}, % ---------| Frame color.
  showspaces=false, % ----------------| Show spaces with underscores.
  showstringspaces=false, % ----------| Underline spaces within strings.
  showtabs=false, % ------------------| Show tabs with underscores.
  stepnumber=1, %---------------------| Step between two line-numbers..
  stringstyle=\color{red}, % ---------| String literal style.
  tabsize=4, % -----------------------| Sets default tabsize to 2 spaces.
  title=\lstname % -------------------| Show the filename of included file.
}
%******************************************************************************
% This preabmle contains packages needed to create figures and plots in LaTeX.
%******************************************************************************

\usepackage{etex}
\usepackage{tikz,pgfplots}
\pgfplotsset{compat=1.9}
\usetikzlibrary{arrows,decorations.markings}
\usetikzlibrary{calc}

% Vector Styles for drawings
\tikzstyle{load}   = [ultra thick,-latex]
\tikzstyle{stress} = [-latex]
\tikzstyle{dim}    = [latex-latex]
\tikzstyle{axis}   = [-latex,black!55]

%******************************************************************************
% Custom environment definitions.
% REQUIRES the `color' package.
%******************************************************************************

\usepackage[amsmath,thmmarks,framed]{ntheorem} % Used to define environments
\usepackage{framed} % -------------------------| Used for framed environments
\usepackage{pstricks} % -----------------------| Used for shaded environments
\definecolor{env-gray}{rgb}{0.9,0.9,0.9}

% Examples environment
\theoremstyle{break}
\theoremframepreskip{1em}
\theoremframepostskip{1.5em}
\theoreminframepreskip{0.8em}
\theoreminframepostskip{0.8em}
\newframedtheorem{example}{\protect\examplename}[section]

% Remarks environment
\theoremstyle{plain}
\theorempreskip{1.5em}
\theorempostskip{1.5em}
\newtheorem{remark}{\protect\remarkname}[section]

% Definitions environment
\theoremstyle{plain}
\theoremframepreskip{\topsep}
\theoremframepostskip{\topsep}
\theoreminframepreskip{0.8em}
\theoreminframepostskip{0.8em}
\theoremindent=1em
\theoremrightindent=1em
\def\theoremframecommand{\colorbox{env-gray}}
\newshadedtheorem{definition}{\protect\definitionname}

% Question environment
\theoremstyle{nonumberbreak}
\theoremframepreskip{\topsep}
\theoremframepostskip{\topsep}
\theoreminframepreskip{0.8em}
\theoreminframepostskip{0.8em}
\theoremindent=1em
\theoremrightindent=1em
\def\theoremframecommand{\colorbox{env-gray}}
\newshadedtheorem{question}{\protect\questionname}

% Program output environment
\theoremstyle{nonumberbreak}
\theoremframepreskip{\topsep}
\theoremframepostskip{\topsep}
\theoreminframepreskip{0.8em}
\theoreminframepostskip{0.8em}
\theoremindent=1em
\theoremrightindent=1em
\def\theoremframecommand{\colorbox{env-gray}}
\newshadedtheorem{programoutput}{\protect\programoutputname}


%---------------------------------
% Translations
%---------------------------------
\addto\captionsenglish{\renewcommand{\corollaryname}{Corollary}}
\addto\captionsenglish{\renewcommand{\definitionname}{Definition}}
\addto\captionsenglish{\renewcommand{\examplename}{Example}}
\addto\captionsenglish{\renewcommand{\remarkname}{Remark}}
\addto\captionsenglish{\renewcommand{\questionname}{Question}}
\addto\captionsenglish{\renewcommand{\programoutputname}{Output}}
\addto\captionsnorsk{\renewcommand{\corollaryname}{Korollar}}
\addto\captionsnorsk{\renewcommand{\definitionname}{Definisjon}}
\addto\captionsnorsk{\renewcommand{\examplename}{Eksempel}}
\addto\captionsnorsk{\renewcommand{\remarkname}{Merknad}}
\addto\captionsnorsk{\renewcommand{\questionname}{Spørsmål}}
\addto\captionsnorsk{\renewcommand{\programoutputname}{Output}}
\providecommand{\corollaryname}{Korollar}
\providecommand{\definitionname}{Definisjon}
\providecommand{\examplename}{Eksempel}
\providecommand{\remarkname}{Merknad}
\providecommand{\questionname}{Spørsmål}
\providecommand{\programoutputname}{Output}


\author{Einar Baumann}
\title{
    \vspace{-1in}
    \usefont{OT1}{bch}{b}{n}
    \vspace{0.1in}
    \rule{\textwidth}{0.5pt} \\[0.5cm]
    \normalfont \normalsize \textsc{TKT4140 Numeriske beregninger med datalab} \\ [20pt]
    {\textsc{ \huge Pythonøving 1}} \\
    \vspace{0.1in}
    \rule{\textwidth}{2pt} \\[0.7cm]
}

\begin{document}
\maketitle
\thispagestyle{empty}
\clearpage

\section{Enkel bruk av Python med numpy} % (fold)
\label{sec:enkel_bruk_av_numpy}


\subsection{Hello world} % (fold)
\label{sub:hello_world}
\lstinputlisting[%
  caption={1a-helloworld.py},
  label={lst:1a},
  language={Python}]
  {../code/1a-helloworld.py}
% subsection hello_world (end)


\subsection{Matriseoperasjoner med numpy} % (fold)
\label{sub:matriseoperasjoner_med_numpy}
Bruker \emph{numpy}-funksjoner til alt her. På linje 6 og 7 brukes \texttt{reshape(m,n)} funksjonen til å omforme et lenært array til en $3\times 3$-matrise.
\lstinputlisting[%
  caption={1b-matops.py},
  label={lst:1b},
  language={Python}]
  {../code/1b-matops.py}
% subsection matriseoperasjoner_med_numpy (end)


\subsection{Skrive ut fibonaccirekken} % (fold)
\label{sub:skrive_ut_fibonaccirekken}
Printer de $n$ første tallene i fibonaccirekken. \texttt{print}-funksjonen fra \emph{Python 3} er importert for å forenkle printing til konsoll uten linjeskift.
\lstinputlisting[%
  caption={1c-fibn.py},
  label={lst:1c},
  language={Python}]
  {../code/1c-fibn.py}
\vspace{-1.2em}
\begin{programoutput}
0 + 1 + 1 + 2 + 3 + 5 + 8 + 13 + 21 + ...
\end{programoutput}
% subsection skrive_ut_fibonaccirekken (end)


\clearpage
\subsection{Sinusplott} % (fold)
\label{sub:sinusplott}
Plottet er vist i Figur~\ref{fig:1d}. Linjene 4-6 endrer fonten muliggjør bruk av LaTeX i plottet. Linje 18 definerer en spesiell type striplet linje. Linje 21-22 spesifiserer spesielle ``ticks'' for $x$-aksen.
\lstinputlisting[%
  caption={1d-sineplot.py},
  label={lst:1d},
  language={Python}]
  {../code/1d-sinplot.py}

\begin{figure}[htbp]
  \centering
  \includegraphics[width=\textwidth]{../code/1d-sineplot.pdf}
  \caption{Sinusplott generert av koden i Listing~\ref{lst:1d}}
  \label{fig:1d}
\end{figure}
% subsection sinusplott (end)



\clearpage
\subsection{Fibonacciplott} % (fold)
\label{sub:fibonacciplott}
Plottet er vist i Figur~\ref{fig:1e}. Funksjonen \texttt{fib(n)} returnerer et array som inneholder de $n$ første fibonaccitallene. På linje 16 genereres et array der hvert element er summen av alle fibonaccitallene opp til det punktet (dårlig forklaring).
\lstinputlisting[%
  caption={1e-fibplot.py},
  label={lst:1e},
  language={Python}]
  {../code/1e-fibplot.py}

\begin{figure}[htbp]
  \centering
  \includegraphics[width=\textwidth]{../code/1e-fibplot.pdf}
  \caption{Plott av summen av fibonaccitallene opp til tall $n$. Generert med koden i Listing~\ref{lst:1e}.}
  \label{fig:1e}
\end{figure}
% subsection fibonacciplott (end)


% section enkel_bruk_av_numpy (end)


\clearpage
\section{Pendel} % (fold)
\label{sec:pendel}
Eulerskjema:
\begin{subequations}
\begin{align}
  \theta_{n+1}      &= \theta_n  + \Delta t \dot\theta_n \\
  \dot\theta_{n+1}  &= \dot\theta_n + \Delta t \ddot\theta_n \\
  \ddot\theta_{n+1} &= - \sin(\theta)
\end{align}
\end{subequations}
Startverdier:
\begin{equation}
  \nonumber
  \theta_0 = \pi/3, \; \dot\theta_0 = + \; \ddot\theta_0 = -\sin(\theta_0)
\end{equation}

\noindent Analytisk løsning, (G.2.14) i komendiet:
\begin{subequations}
\begin{align}
  \theta     &= 2 \arcsin \left( k\cdot \mathrm{sn}(T-t,k) \right) \\
  \dot\theta &= 2 k \cdot \mathrm{cn}(T-t,k)
\end{align}
\end{subequations}

Løsningen med Eulers metode er selvforklarende. Få å få de analytiske verdiene brukte jeg \texttt{ellipk} og \texttt{ellipj} funksjonene fra \texttt{scipy.special} modulen. Firkemåten til \texttt{ellipk} er enkel: den returnerer bare ett tall. \texttt{ellipj} derimot returnerer verdien for flere forskjellige relaterte funksjoner: \textbf{sn}, \textbf{cn}, \textbf{dn} og \textbf{ph}; av disse er det bare \textbf{sn} og \textbf{cn} vi faktisk bruker.

Plottene av utslaget og energifunksjonen er vist i hhv. Figur \ref{fig:2_theta} og \ref{fig:2_E}.

\lstinputlisting[%
  caption={2-pendel.py},
  label={lst:2},
  language={Python}]
  {../code/2-pendel.py}

\begin{figure}[H]
  \centering
  \includegraphics[width=0.8\textwidth]{../code/pendel_theta.pdf}
  \caption{Plott av vinkelutslaget $\theta$ over tid. Generert med koden i Listing~\ref{lst:2}.}
  \label{fig:2_theta}
\end{figure}

\begin{figure}[H]
  \centering
  \includegraphics[width=0.8\textwidth]{../code/pendel_E.pdf}
  \caption{Plott av energifunksjonen til pendelen. Generert med koden i Listing~\ref{lst:2}.}
  \label{fig:2_E}
\end{figure}

% section pendel (end)



\end{document}
