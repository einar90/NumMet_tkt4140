\documentclass[norsk,11pt,a4paper]{article}
\usepackage[T1]{fontenc} % --------------| More characters.
\usepackage[utf8]{inputenc} % ---------| Direct use of scandinavian letters.
\usepackage{float} % --------------------| More options for floats.
\usepackage{graphicx} % -----------------| Support more image formats.
\usepackage{booktabs} % -----------------| Better-looking tables.
\usepackage{tabularx} % -----------------| Better tables
\usepackage{subfig} % -------------------| Subfigures.
\usepackage[a4paper]{geometry} % --------| Adjusting page margins.
\usepackage{amsmath,amssymb,amsfonts} % -| Various math, including eqref.
\usepackage{color} % --------------------| Allows defn. of custom colors.
\usepackage{babel}

% XY-pic. Used for creating illustrations.
\input xy
\xyoption{all}

% Styling captions.
\usepackage{caption}
\captionsetup{margin=10pt,font=small,labelfont=bf}

% Changing section header and title fonts.
\usepackage{sectsty}
\allsectionsfont{\normalfont\sffamily}


%******************************************************************************
% Custom environment definitions.
% REQUIRES the `color' package.
%******************************************************************************

\usepackage[amsmath,thmmarks,framed]{ntheorem} % Used to define environments
\usepackage{framed} % -------------------------| Used for framed environments
\usepackage{pstricks} % -----------------------| Used for shaded environments
\definecolor{env-gray}{rgb}{0.9,0.9,0.9}

% Examples environment
\theoremstyle{break}
\theoremframepreskip{1em}
\theoremframepostskip{1.5em}
\theoreminframepreskip{0.8em}
\theoreminframepostskip{0.8em}
\newframedtheorem{example}{\protect\examplename}[section]

% Remarks environment
\theoremstyle{plain}
\theorempreskip{1.5em}
\theorempostskip{1.5em}
\newtheorem{remark}{\protect\remarkname}[section]

% Definitions environment
\theoremstyle{plain}
\theoremframepreskip{\topsep}
\theoremframepostskip{\topsep}
\theoreminframepreskip{0.8em}
\theoreminframepostskip{0.8em}
\theoremindent=1em
\theoremrightindent=1em
\def\theoremframecommand{\colorbox{env-gray}}
\newshadedtheorem{definition}{\protect\definitionname}

% Question environment
\theoremstyle{nonumberbreak}
\theoremframepreskip{\topsep}
\theoremframepostskip{\topsep}
\theoreminframepreskip{0.8em}
\theoreminframepostskip{0.8em}
\theoremindent=1em
\theoremrightindent=1em
\def\theoremframecommand{\colorbox{env-gray}}
\newshadedtheorem{question}{\protect\questionname}

% Program output environment
\theoremstyle{nonumberbreak}
\theoremframepreskip{\topsep}
\theoremframepostskip{\topsep}
\theoreminframepreskip{0.8em}
\theoreminframepostskip{0.8em}
\theoremindent=1em
\theoremrightindent=1em
\def\theoremframecommand{\colorbox{env-gray}}
\newshadedtheorem{programoutput}{\protect\programoutputname}


%---------------------------------
% Translations
%---------------------------------
\addto\captionsenglish{\renewcommand{\corollaryname}{Corollary}}
\addto\captionsenglish{\renewcommand{\definitionname}{Definition}}
\addto\captionsenglish{\renewcommand{\examplename}{Example}}
\addto\captionsenglish{\renewcommand{\remarkname}{Remark}}
\addto\captionsenglish{\renewcommand{\questionname}{Question}}
\addto\captionsenglish{\renewcommand{\programoutputname}{Output}}
\addto\captionsnorsk{\renewcommand{\corollaryname}{Korollar}}
\addto\captionsnorsk{\renewcommand{\definitionname}{Definisjon}}
\addto\captionsnorsk{\renewcommand{\examplename}{Eksempel}}
\addto\captionsnorsk{\renewcommand{\remarkname}{Merknad}}
\addto\captionsnorsk{\renewcommand{\questionname}{Spørsmål}}
\addto\captionsnorsk{\renewcommand{\programoutputname}{Output}}
\providecommand{\corollaryname}{Korollar}
\providecommand{\definitionname}{Definisjon}
\providecommand{\examplename}{Eksempel}
\providecommand{\remarkname}{Merknad}
\providecommand{\questionname}{Spørsmål}
\providecommand{\programoutputname}{Output}

%******************************************************************************
% Includes the listings package and sets some settings for it.
% REQUIRES the `color' package.
%******************************************************************************

\usepackage{listings} % -----------------------| Used for source code listings.
\definecolor{lst-gray}{RGB}{100,100,100}  % ---|Color for line-numbers.
\definecolor{lst-light-gray}{RGB}{250,250,250} % Background color for listings.
\definecolor{lst-dark-green}{RGB}{45,111,0} % -| Dark green for comments.
\lstset{
  aboveskip=0em, % -------------------| Skip above listing box.
  backgroundcolor=\color{lst-light-gray}, % Background color.
  basicstyle=\ttfamily\scriptsize, % -| Default font style.
  belowskip=\topskip, % --------------| Skip below listing box.
  breakatwhitespace=false, % ---------| Automatic breaks only at whitespace?.
  breaklines=true, % -----------------| Sets automatic line breaking.
  captionpos=t, % --------------------| Sets the caption-position to bottom.
  commentstyle=\color{lst-dark-green}, % ------| Comment style.
  escapeinside={\%*}{*)}, % ----------| For adding LaTeX within code.
  frame=single, % --------------------| Adds a frame around the code.
  keepspaces=true, % -----------------| Keeps spaces in text.
  keywordstyle=\color{blue}, %--------| Keyword style.
  language={C}, % --------------------| The language of the code.
  literate={æ}{{\ae}}1 % -------------| Character conversions
           {Æ}{{\AE}}1
           {ø}{{\oe}}1
           {Ø}{{\OE}}1
           {å}{{\aa}}1
           {Å}{{\AA}}1
           {µ}{{\ensuremath{\mu}}}1,
  numbers=left, % --------------------| Line-number position: none/left/right.
  numbersep=5pt, % -------------------| Distance between line-numbers and code.
  numberstyle=\tiny\color{lst-gray}, %| The style that used for line-numbers.
  rulecolor=\color{black}, % ---------| Frame color.
  showspaces=false, % ----------------| Show spaces with underscores.
  showstringspaces=false, % ----------| Underline spaces within strings.
  showtabs=false, % ------------------| Show tabs with underscores.
  stepnumber=1, %---------------------| Step between two line-numbers..
  stringstyle=\color{red}, % ---------| String literal style.
  tabsize=4, % -----------------------| Sets default tabsize to 2 spaces.
  title=\lstname % -------------------| Show the filename of included file.
}
%******************************************************************************
% This preabmle contains packages needed to create figures and plots in LaTeX.
%******************************************************************************

\usepackage{etex}
\usepackage{tikz,pgfplots}
\pgfplotsset{compat=1.9}
\usetikzlibrary{arrows,decorations.markings}
\usetikzlibrary{calc}

% Vector Styles for drawings
\tikzstyle{load}   = [ultra thick,-latex]
\tikzstyle{stress} = [-latex]
\tikzstyle{dim}    = [latex-latex]
\tikzstyle{axis}   = [-latex,black!55]



\author{Einar Baumann}
\title{
    \vspace{-1in}
    \usefont{OT1}{bch}{b}{n}
    \vspace{0.1in}
    \rule{\textwidth}{0.5pt} \\[0.5cm]
    \normalfont \normalsize \textsc{TKT4140 Numeriske beregninger med datalab} \\ [20pt]
    {\textsc{ \huge Matlabøving 3 }} \\ [0.5cm]
    {\textsc {\Large Jeffrey-Hamel strømning og eksempel fra 2.2} } \\
    \vspace{0.1in}
    \rule{\textwidth}{2pt} \\[0.7cm]
}

\begin{document}
\maketitle
\thispagestyle{empty}
\clearpage

\section{Jeffrey-Hamel strømning} % (fold)
\label{sec:jeffrey_hamel_str_mning}
Ligningen som løses er som følger:
\begin{equation}
  f'''(\eta) + 2 \mathrm{Re} \alpha \cdot f(\eta) \cdot f'(eta) + 4 \alpha^2 f'(\eta) = 0
\end{equation}
\begin{equation}
  \tag{b}
  f(0) = 1, \;\; f'(0) = 0, \;\; f(1) = 0
\end{equation}
Skrevet om til ligningssystem:
\begin{subequations}
\begin{align}
  f'_1 &= f_2 \\
  f'_2 &= f_3 \\
  f'_3 &= -2 \alpha \cdot f_2 \left( \mathrm{Re}\cdot f_1 2 \alpha \right)
\end{align}
\end{subequations}
\begin{equation}
  \tag{b}
  f_1(0) = 1, \;\; f_2(0) = 0, \;\; f_1(1) = 0
\end{equation}

Dette ligningssystemet er programmert i funksjonen vist i Listing~\ref{lst:jefham}. Skytemetoden med sekantmetoden er brukt til å finne $s^*$ og løse systemet. Dette er vist i Listing~\ref{lst:jefhamsec}. Output av programmet er som følger:

\begin{programoutput}
\begin{verbatim}
        itr.      s          ds

         1    -6.456906   -4.569e-01
         2    -6.485060   -2.815e-02
         3    -6.483644    1.416e-03
         4    -6.483648   -4.056e-06

         eta        f          f'        f"

         0.00   1.000000   0.000000  -6.48364e+00
         1.00   0.000001  -0.078316   3.33591e+00
\end{verbatim}
\end{programoutput}
Vi har altså at $s^* = y''(0) \approx -6.484$. De beregnede verdiene for $f(\eta), f'(\eta), f''(\eta)$ med denne initialverdien er vist i Figur~\ref{fig:jefham}.

\clearpage

\begin{figure}[htbp]
  \centering
  \includegraphics[width=0.95\textwidth]{graphics/plot-jefham}
  \caption{Plott av løsningen til Jeffrey-Hamel ligningen}
  \label{fig:jefham}
\end{figure}

\lstinputlisting[%
  caption={jefham.m},
  label={lst:jefham},
  language={Matlab}]
  {../code/jefham.m}

\lstinputlisting[%
  caption={jefhamsec.m},
  label={lst:jefhamsec},
  language={Matlab}]
  {../code/jefhamsec.m}

% section jeffrey_hamel_str_mning (end)


\clearpage
\section{Eksempel fra seksjon 2.2} % (fold)
\label{sec:eksempel_fra_seksjon_2_2}
Ligningen løsses er som følger:
\begin{equation}
  y''(x) = \frac{3}{2} \left[ y(x) \right]^2
\end{equation}
\begin{equation}
  \tag{b}
  y'(0) = -8, \;\; y'(1) = -1
\end{equation}
Skrevet om til ligningssystem:
\begin{subequations}
\begin{align}
  y'_1(x) &= y_2(x) \\
  y'_2(x) &= \frac{3}{2} \left[ y_1(x) \right]^2
\end{align}
\end{subequations}
\begin{equation}
  \tag{b}
  y_2(0) = -8, \;\; y_2(1) = -1
\end{equation}

Dette ligningssystemet er programmert i funksjonen vist i Listing~\ref{lst:eks22fun}. Skytemetoden med sekantmetoden er brukt til å finne $s^*$ og løse systemet. Dette er vist i Listing~\ref{lst:eks_avs2_2.m}. Output av programmet er som følger:

\begin{programoutput}
\begin{verbatim}
        itr.      s          ds

         1     2.369604   -6.304e-01
         2     2.354057   -1.555e-02
         3     2.355254    1.197e-03
         4     2.355254    6.658e-08

         x        y          y'

         0.00   2.355254  -8.000000
         1.00  -3.682433  -1.000000
\end{verbatim}
\end{programoutput}

Vi har altså at $s^* = y(0) \approx 2.355$. De beregnede verdiene for $f(x)$ og $f'(x)$ med denne initialverdien er vist i Figur~\ref{fig:eks22}.
\clearpage

\begin{figure}[htbp]
  \centering
  \includegraphics[width=0.95\textwidth]{graphics/plot-eks22}
  \caption{Plott av løsning til ligningen i eksempel i seksjon 2.2.}
  \label{fig:eks22}
\end{figure}

\lstinputlisting[%
  caption={eks22fun.m},
  label={lst:eks22fun},
  language={Matlab}]
  {../code/eks22fun.m}

\lstinputlisting[%
  caption={eks\_avs2\_2.m},
  label={lst:eks_avs2_2.m},
  language={Matlab}]
  {../code/eks_avs2_2.m}

% section eksempel_fra_seksjon_2_2 (end)



\end{document}
