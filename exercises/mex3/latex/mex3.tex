\input{../../../common/preamble.tex}
\input{../../../common/preamble-addon-envs.tex}
\input{../../../common/preamble-addon-listings.tex}
\input{../../../common/preamble-addon-figures.tex}


\author{Einar Baumann}
\title{
    \vspace{-1in}
    \usefont{OT1}{bch}{b}{n}
    \vspace{0.1in}
    \rule{\textwidth}{0.5pt} \\[0.5cm]
    \normalfont \normalsize \textsc{TKT4140 Numeriske beregninger med datalab} \\ [20pt]
    {\textsc{ \huge Matlabøving 3 }} \\ [0.5cm]
    {\textsc {\Large Jeffrey-Hamel strømning og eksempel fra 2.2} } \\
    \vspace{0.1in}
    \rule{\textwidth}{2pt} \\[0.7cm]
}

\begin{document}
\maketitle
\thispagestyle{empty}
\clearpage

\section{Jeffrey-Hamel strømning} % (fold)
\label{sec:jeffrey_hamel_str_mning}
Ligningen som løses er som følger:
\begin{equation}
  f'''(\eta) + 2 \mathrm{Re} \alpha \cdot f(\eta) \cdot f'(eta) + 4 \alpha^2 f'(\eta) = 0
\end{equation}
\begin{equation}
  \tag{b}
  f(0) = 1, \;\; f'(0) = 0, \;\; f(1) = 0
\end{equation}
Skrevet om til ligningssystem:
\begin{subequations}
\begin{align}
  f'_1 &= f_2 \\
  f'_2 &= f_3 \\
  f'_3 &= -2 \alpha \cdot f_2 \left( \mathrm{Re}\cdot f_1 2 \alpha \right)
\end{align}
\end{subequations}
\begin{equation}
  \tag{b}
  f_1(0) = 1, \;\; f_2(0) = 0, \;\; f_1(1) = 0
\end{equation}

Dette ligningssystemet er programmert i funksjonen vist i Listing~\ref{lst:jefham}. Skytemetoden med sekantmetoden er brukt til å finne $s^*$ og løse systemet. Dette er vist i Listing~\ref{lst:jefhamsec}. Output av programmet er som følger:

\begin{programoutput}
\begin{verbatim}
        itr.      s          ds

         1    -6.456906   -4.569e-01
         2    -6.485060   -2.815e-02
         3    -6.483644    1.416e-03
         4    -6.483648   -4.056e-06

         eta        f          f'        f"

         0.00   1.000000   0.000000  -6.48364e+00
         1.00   0.000001  -0.078316   3.33591e+00
\end{verbatim}
\end{programoutput}
Vi har altså at $s^* = y''(0) \approx -6.484$. De beregnede verdiene for $f(\eta), f'(\eta), f''(\eta)$ med denne initialverdien er vist i Figur~\ref{fig:jefham}.

\clearpage

\begin{figure}[htbp]
  \centering
  \includegraphics[width=0.95\textwidth]{graphics/plot-jefham}
  \caption{Plott av løsningen til Jeffrey-Hamel ligningen}
  \label{fig:jefham}
\end{figure}

\lstinputlisting[%
  caption={jefham.m},
  label={lst:jefham},
  language={Matlab}]
  {../code/jefham.m}

\lstinputlisting[%
  caption={jefhamsec.m},
  label={lst:jefhamsec},
  language={Matlab}]
  {../code/jefhamsec.m}

% section jeffrey_hamel_str_mning (end)


\clearpage
\section{Eksempel fra seksjon 2.2} % (fold)
\label{sec:eksempel_fra_seksjon_2_2}
Ligningen løsses er som følger:
\begin{equation}
  y''(x) = \frac{3}{2} \left[ y(x) \right]^2
\end{equation}
\begin{equation}
  \tag{b}
  y'(0) = -8, \;\; y'(1) = -1
\end{equation}
Skrevet om til ligningssystem:
\begin{subequations}
\begin{align}
  y'_1(x) &= y_2(x) \\
  y'_2(x) &= \frac{3}{2} \left[ y_1(x) \right]^2
\end{align}
\end{subequations}
\begin{equation}
  \tag{b}
  y_2(0) = -8, \;\; y_2(1) = -1
\end{equation}

Dette ligningssystemet er programmert i funksjonen vist i Listing~\ref{lst:eks22fun}. Skytemetoden med sekantmetoden er brukt til å finne $s^*$ og løse systemet. Dette er vist i Listing~\ref{lst:eks_avs2_2.m}. Output av programmet er som følger:

\begin{programoutput}
\begin{verbatim}
        itr.      s          ds

         1     2.369604   -6.304e-01
         2     2.354057   -1.555e-02
         3     2.355254    1.197e-03
         4     2.355254    6.658e-08

         x        y          y'

         0.00   2.355254  -8.000000
         1.00  -3.682433  -1.000000
\end{verbatim}
\end{programoutput}

Vi har altså at $s^* = y(0) \approx 2.355$. De beregnede verdiene for $f(x)$ og $f'(x)$ med denne initialverdien er vist i Figur~\ref{fig:eks22}.
\clearpage

\begin{figure}[htbp]
  \centering
  \includegraphics[width=0.95\textwidth]{graphics/plot-eks22}
  \caption{Plott av løsning til ligningen i eksempel i seksjon 2.2.}
  \label{fig:eks22}
\end{figure}

\lstinputlisting[%
  caption={eks22fun.m},
  label={lst:eks22fun},
  language={Matlab}]
  {../code/eks22fun.m}

\lstinputlisting[%
  caption={eks\_avs2\_2.m},
  label={lst:eks_avs2_2.m},
  language={Matlab}]
  {../code/eks_avs2_2.m}

% section eksempel_fra_seksjon_2_2 (end)



\end{document}
