\usepackage[T1]{fontenc}
\usepackage{droid}
\usepackage[defaultsans]{opensans}
\usepackage{listings}

\usepackage{sectsty}
\allsectionsfont{\normalfont\sffamily}

\usepackage[usenames,dvipsnames]{color}   % Used for colors by various other packages
\definecolor{light-gray}{RGB}{240,240,240}  % Background color for listings
\definecolor{gray}{RGB}{100,100,100}  % Color for line-numbers
\lstset{
  language={C},     % the language of the code
  aboveskip=0em,
  belowskip=-3em,
  basicstyle=\ttfamily\scriptsize,
  backgroundcolor=\color{light-gray},
  frame=single,       % adds a frame around the code
  rulecolor=\color{black},  % frame color
  captionpos=t,       % sets the caption-position to bottom
  breaklines=true,      % sets automatic line breaking
  breakatwhitespace=false,  % automatic breaks only at whitespace?
  keepspaces=true,      % keeps spaces in text, useful for keeping indentation of code (possibly needs columns=flexible)
  tabsize=4,          % sets default tabsize to 2 spaces
  commentstyle=\color{green}, % comment style
  keywordstyle=\color{blue},  % keyword style
  stringstyle=\color{red},  % string literal style
  numbers=left,       % line-number position: none, left or right
  numbersep=5pt,        % how far the line-numbers are from the code
  numberstyle=\tiny\color{gray}, % the style that used for line-numbers
  stepnumber=2,       % step between two line-numbers.
  showspaces=false,     % show spaces everywhere with underscores
  showtabs=false,       % show tabs within strings with underscores
  showstringspaces=false,   % underline spaces within strings only
  escapeinside={\%*}{*)},   % if you want to add LaTeX within your code
  title=\lstname        % show the filename of files included with \lstinputlisting; also try caption instead of title
}
